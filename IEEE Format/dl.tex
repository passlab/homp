\section{Data layout/access in the memory systems Graph}
Data Layout in memory systems graph shows how the data is allocated in the NUMA region. Uneven distribution of memory accesses to NUMA region often lead to contention and unnecessary bandwidth saturation in both off-chip and on-chip interconnects, memory controllers and caches. Solution can be, instead of mapping large data objects to a single NUMA domain, in many cases one can reduce associated bandwidth saturation and contention by distributing large data objects across all NUMA regions. This is called as optimization contention reduction. Data layout in memory system graph also tells how the data is moving vertically across the memory hierarchy and horizontally across the cache at the same level. It also allows user to understand mapping between multiple discrete memory spaces such as in accelerators. This gives user the information of the page suffering from uneven memory requests, so that one can use various allocation methods to balance the memory requests. The Data layout/access in the memory systems Graph will also allow analysis of the access patterns across the threads to guide NUMA locality optimization, and thereby identify where and which data pages are bounded to NUMA region. User can visualize the shared data access by multiple cores in the system. The visualization tool also show the mapping of virtual pages to the physical pages at the runtime as shown in fig. 3. It shows the mapping of matrix A across NUMA node 0 and NUMA node1. The graph also allows to visualize the pages allocated on each NUMA node as shown in fig. 4, helping user to bind the pages to specific memory location to reduce contention and false sharing.
\begin{figure}[!t]
\centering
\includegraphics [width=2.5in] {"data layout".png}
\caption{Page mapping on NUMA node 0}
\label{fig4}
\end{figure}